\documentclass[12pt]{article}

\usepackage[czech]{babel}
\usepackage{a4wide}
\usepackage[utf8]{inputenc}
\usepackage[T1]{fontenc}
\usepackage{blindtext}
\usepackage{fancyhdr}
\usepackage{amssymb}
\usepackage{amsthm}
\usepackage{amsmath}
\usepackage{mathtools}
\usepackage{mleftright}
\usepackage{subfig}

% forcing footnotes to be at the very bottom
\usepackage[bottom]{footmisc}

\usepackage{hyperref}
\usepackage{titlesec}
%This has to be the last
\usepackage{subfiles}

\usepackage{geometry}
\geometry{
    a4paper,
    right=20mm,
    left=20mm,
    top=30mm,
    bottom=20mm,
}

\DeclareMathOperator{\rank}{rank}
\DeclareMathOperator{\Span}{span}

\newtheoremstyle{nontheoremstyle}{1em}{1em}{}{}{\bfseries}{:}{.5em}{}
\newtheoremstyle{theoremstyle}{1em}{1em}{\it}{}{\bfseries}{:}{.5em}{}

\theoremstyle{nontheoremstyle}
\newtheorem*{definition}{Definice}
\newtheorem*{example}{Příklad}
\renewenvironment{proof}{{\noindent\bfseries Důkaz:}}{\qed}
\newtheorem*{intuition}{Intuice}
\newtheorem*{remark}{Poznámka}
\newtheorem*{consequence}{Důsledek}
\newtheorem*{observation}{Pozorování}

% ošklivý hack k tomu, aby environment 'definitionnodot' neměl na konci . nebo :
% hodí se, když chceme Dělat něco jako „Definice (Riemannův integrál) je funkce...“
\newtheoremstyle{nontheoremstylenodot}{1em}{1em}{}{}{\bfseries}{}{.3em}{}
\theoremstyle{nontheoremstylenodot}
\newtheorem*{definitionnodot}{Definice}

\theoremstyle{theoremstyle}
\newtheorem*{theorem}{Věta}
\newtheorem*{lemma}{Tvrzení}

\titleformat{\part} {\normalfont\fontsize{21}{21}\bfseries}{\thepart}{1em}{}
\titleformat{\section} {\normalfont\fontsize{17}{15}\bfseries}{\thesection}{1em}{}
\titleformat{\subsection} {\normalfont\fontsize{14}{15}\bfseries}{\thesubsection}{1em}{}
\titleformat{\subsubsection} {\normalfont\fontsize{12}{15}\bfseries}{\thesubsubsection}{1em}{}

\setlength{\headheight}{16pt}
\pagestyle{fancy}
\fancyhf{}
\lhead{\leftmark}
\rhead{Poznámky ke státnicím}
\fancyfoot{}
\fancyfoot[R]{\thepage}

\begin{document}

\begin{titlepage}
    \begin{center}
        \vspace*{1cm}
            
        \Huge
        \textbf{Poznámky ke státnícím}


        \vspace{3mm}
        Letní semestr 2022/2023
        
        \vspace{1.5cm}
            
        \textbf{Viktor Soukup, Lukáš Salak}
        
        \vfill
        \flushright
        \normalsize
        \today
        
    \end{center}
\end{titlepage}

\tableofcontents
%\clearpage


\part{Matematika}

\subfile{Sections/Matematika/01-zaklady-diferencialniho-a-integralniho-poctu.tex}
\subfile{Sections/Matematika/02-algebra-a-linearni-algebra.tex}
\subfile{Sections/Matematika/03-diskretni-matematika.tex}
\subfile{Sections/Matematika/04-teorie-grafu.tex}
\subfile{Sections/Matematika/05-pravdepodobnost-a-statistika.tex}
\subfile{Sections/Matematika/06-logika.tex}

\part{Informatika}

\subfile{Sections/Informatika/01-automaty-a-jazyky.tex}
\subfile{Sections/Informatika/02-algoritmy-a-datove-struktury.tex}
\subfile{Sections/Informatika/03-programovaci-jazyky.tex}
\subfile{Sections/Informatika/04-architektura-pocitacu-a-operacnich-systemu.tex}

\part{Specializace 6 - Počítačová grafika, vidění a vývoj her}

\subfile{Sections/Spec06-Grafika/01-matematicka-analyza.tex}
\subfile{Sections/Spec06-Grafika/02-zaklady-2d-pocitacove-grafiky.tex}
\subfile{Sections/Spec06-Grafika/03-zaklady-3d-pocitacove-grafiky.tex}
\subfile{Sections/Spec06-Grafika/04-fotorealisticka-grafika.tex}
\subfile{Sections/Spec06-Grafika/05-zaklady-vedy-o-barvach.tex}
\subfile{Sections/Spec06-Grafika/06-geometrie-pro-pocitacovou-grafiku.tex}

\vfill
\begin{center}
\LARGE
\textbf{The End}
\end{center}

\end{document}